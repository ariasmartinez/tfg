\chapter*{}
%\thispagestyle{empty}
%\cleardoublepage

%\thispagestyle{empty}

%\input{portada/portada_2}



\cleardoublepage
\thispagestyle{empty}

\begin{center}
{\large\bfseries Sistema de adaptación motora con entorno de realidad virtual}\\
\end{center}
\begin{center}
Celia Arias Martínez\\
\end{center}

%\vspace{0.7cm}
\noindent{\textbf{Palabras clave}: realidad virtual, dispositivos hápticos, aprendizaje de movimientos.}\\

\vspace{0.7cm}
\noindent{\textbf{Resumen}}\\

Los dispositivos de realidad virtual son una herramienta clave en el estudio de determinar los procesos que intervienen en la realización y aprendizaje de determinados movimientos. Los dispositivos hápticos pueden ayudarnos en dicha tarea, ya que proporcionan otra capa más de inmersión dentro del entorno de realidad virtual que queremos recrear. Con este proyecto estudiaremos cómo se puede integrar un dispositivo háptico dentro de una aplicación que se utilizará para analizar en diferentes sujetos los procesos de aprendizaje de unos determinados movimientos. De esta forma queremos estudiar si los dispositivos hápticos pueden ser una herramienta más en la realización de este tipo de experimentos, así como los problemas que podríamos encontrar al trabajar con ellos.


\cleardoublepage


\thispagestyle{empty}


\begin{center}
{\large\bfseries Project Title: Project Subtitle}\\
\end{center}
\begin{center}
Celia Arias Martínez\\
\end{center}

%\vspace{0.7cm}
\noindent{\textbf{Keywords}: virtual reality, haptic display, movement learning}\\

\vspace{0.7cm}
\noindent{\textbf{Abstract}}\\



\chapter*{}
\thispagestyle{empty}

\noindent\rule[-1ex]{\textwidth}{2pt}\\[4.5ex]

Yo, \textbf{Nombre Apellido1 Apellido2}, alumno de la titulación TITULACIÓN de la \textbf{Escuela Técnica Superior
de Ingenierías Informática y de Telecomunicación de la Universidad de Granada}, con DNI XXXXXXXXX, autorizo la
ubicación de la siguiente copia de mi Trabajo Fin de Grado en la biblioteca del centro para que pueda ser
consultada por las personas que lo deseen.

\vspace{6cm}

\noindent Fdo: Nombre Apellido1 Apellido2

\vspace{2cm}

\begin{flushright}
Granada a X de mes de 201 .
\end{flushright}


\chapter*{}
\thispagestyle{empty}

\noindent\rule[-1ex]{\textwidth}{2pt}\\[4.5ex]

D. \textbf{Nombre Apellido1 Apellido2 (tutor1)}, Profesor del Área de XXXX del Departamento YYYY de la Universidad de Granada.

\vspace{0.5cm}

D. \textbf{Nombre Apellido1 Apellido2 (tutor2)}, Profesor del Área de XXXX del Departamento YYYY de la Universidad de Granada.


\vspace{0.5cm}

\textbf{Informan:}

\vspace{0.5cm}

Que el presente trabajo, titulado \textit{\textbf{Título del proyecto, Subtítulo del proyecto}},
ha sido realizado bajo su supervisión por \textbf{Nombre Apellido1 Apellido2 (alumno)}, y autorizamos la defensa de dicho trabajo ante el tribunal
que corresponda.

\vspace{0.5cm}

Y para que conste, expiden y firman el presente informe en Granada a X de mes de 201 .

\vspace{1cm}

\textbf{Los directores:}

\vspace{5cm}

\noindent \textbf{Nombre Apellido1 Apellido2 (tutor1) \ \ \ \ \ Nombre Apellido1 Apellido2 (tutor2)}

\chapter*{Agradecimientos}
\thispagestyle{empty}

       \vspace{1cm}


Poner aquí agradecimientos...


\chapter*{}
%\thispagestyle{empty}
%\cleardoublepage

%\thispagestyle{empty}

%\input{portada/portada_2}



\cleardoublepage
\thispagestyle{empty}

\begin{center}
{\large\bfseries Sistema de adaptación motora con entorno de realidad virtual}\\
\end{center}
\begin{center}
Celia Arias Martínez\\
\end{center}

%\vspace{0.7cm}
\noindent{\textbf{Palabras clave}: realidad virtual, dispositivos hápticos, aprendizaje motor.}\\

\vspace{0.7cm}
\noindent{\textbf{Resumen}}\\

Los dispositivos de realidad virtual son una herramienta clave en el estudio de la determinación de los procesos que intervienen en la realización y aprendizaje de determinados movimientos. Los dispositivos hápticos pueden ayudarnos en dicha tarea, ya que proporcionan otra capa más de inmersión dentro del entorno de realidad virtual que queremos recrear. Con este proyecto estudiaremos cómo se puede integrar nuestro dispositivo háptico dentro de una aplicación que se utilizará para analizar los procesos de aprendizaje de unos movimientos específicos. De esta forma, determinaremos si los dispositivos hápticos pueden ser una herramienta más en la realización de este tipo de experimentos, así como los problemas que podríamos encontrar al trabajar con ellos.


\cleardoublepage


\thispagestyle{empty}


\begin{center}
{\large\bfseries Motor adaptation system with virtual reality environment}\\
\end{center}
\begin{center}
Celia Arias Martínez\\
\end{center}

%\vspace{0.7cm}
\noindent{\textbf{Keywords}: virtual reality, haptic display, movement learning.}\\

\vspace{0.7cm}
\noindent{\textbf{Abstract}}\\

Virtual reality devices are a key tool in the study of determining the processes involved in performing and learning certain movements. Haptic devices can help us in this task, as they provide another layer of immersion within the virtual reality environment that we want to recreate. In this project we will study how a haptic device can be integrated into an application that will be used to analyze  the learning processes of certain movements. In this way, we will determine if haptic devices could be another tool in the realization of this type of experiments, as well as the problems that we could find when working with them.

\chapter*{}
\thispagestyle{empty}

\noindent\rule[-1ex]{\textwidth}{2pt}\\[4.5ex]

Yo, \textbf{Celia Arias Martínez}, alumno de la titulación Doble Grado de Ingeniería Informática y Matemáticas de la \textbf{Universidad de Granada}, con DNI 26510285W, autorizo la
ubicación de la siguiente copia de mi Trabajo Fin de Grado en la biblioteca del centro para que pueda ser
consultada por las personas que lo deseen.

\vspace{6cm}

\noindent Fdo: Celia Arias Martínez

\vspace{2cm}

\begin{flushright}
Granada, junio de 2023.
\end{flushright}


\chapter*{}
\thispagestyle{empty}

\noindent\rule[-1ex]{\textwidth}{2pt}\\[4.5ex]

D. \textbf{Eduardo Ros Vidal}, Profesor del Departamento de Ingeniería de Computadores, Automática y Robótica de la Universidad de Granada.

\vspace{0.5cm}

D. \textbf{Jesús Garrido Alcázar}, Profesor  del Departamento de Ingeniería de Computadores, Automática y Robótica de la Universidad de Granada.


\vspace{0.5cm}

\textbf{Informan:}

\vspace{0.5cm}

Que el presente trabajo, titulado \textit{\textbf{Sistema de adaptación de motora con entorno de realidad virtual}},
ha sido realizado bajo su supervisión por \textbf{Celia Arias Martínez}, y autorizamos la defensa de dicho trabajo ante el tribunal
que corresponda.

\vspace{0.5cm}

Y para que conste, expiden y firman el presente informe en Granada a junio de 2023.

\vspace{1cm}

\textbf{Los directores:}

\vspace{5cm}

\noindent \textbf{Eduardo Ros Vidal \ \ \ \ \  Jesús Garrido Alcázar}

\chapter*{Agradecimientos}
\thispagestyle{empty}

       \vspace{1cm}


Poner aquí agradecimientos...

